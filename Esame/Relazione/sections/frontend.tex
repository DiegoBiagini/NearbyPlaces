\section*{Front End}
Il front end si occupa di costruire in modo dinamico tutta la pagina, in base a delle coordinate fornite dall'utente(proprietario del sito web).\\
Abbiamo quindi realizzato una pagina .html ed un file javascript che permette di costruire parte della pagina html in modo dinamico.
\subsection*{Realizzazione pagina statica}
In questa fase abbiamo realizzato la grafica della pagina basandoci sul mockup del progetto e integrandone le classi css di bootstrap, per sfruttare 
l'adattamento della pagina (page responsive).\\ 
SCREEN MOKUP(AGGIORNATO)
\subsection*{Ottenimento informazioni dall'API} 
L'obiettivo principale di questa fase consiste nell'ottenere in base a delle coordinate, passate dal nostro plugin wordpress, dei luoghi suggeriteci dall'API di Wikipedia.\\
Per eseguire questo tipo di azione abbiamo effettuato una chiamata all'API mediante l'utilizzo di jquery.
La struttura della chiamata è composta dai seguenti parametri:
\begin{itemize}
\item \texttt{"action": "query"} modulo che serve per recuperare dei dati dall'API;
\item \texttt{"prop": "coordinates|pageimages|description"} ci permette di specificare di quali informazioni abbiamo bisogno;
\item \texttt{"pithumbsize": 150} dimensione dell'immagine che ci restituirà l'API;
\item \texttt{"generator": "geosearch"} generatore usato per ottenere dei risultati di ricerca per un dato insieme di pagine.
\item \texttt{"ggslimit": "12"} specifica di quanti luoghi la risposta sarà formata;
\item \texttt{"ggscoord":  positions.latitude + "|" + positions.longitude} indica le coordinate che passiamo all'API.
\item \texttt{"format": "json"} specifica che vogliamo una riposta di tipo JSON.
\end{itemize}
\subsection*{Costruzione pagina dinamica} 
Qualora la ricerca effettuata sul servizio di Wikipedia restituisca esito negativo, verrà mostrato il seguente messaggio di errore: \texttt{"Unknown error during the request"}.\\
Qualora invece, la ricerca restituisca esito positivo, otterremo un JSON con tutte le varie informazioni dei luoghi correlati alla posizione scelta dall'utente (proprietario del sito web).\\
La risposta ottenuta è composta da vari vettori JSON.\\
Questi verranno scomposti in modo da ottenere le informazioni che andremo ad aggiungere all'interno del "card-layout" (costruito dinamicamente).\\
A questo punto, il "card-layout" verrà inserito all'interno del carosello bootstrap per consentire una migliore visualizzazione e ottimizzazione della pagina.\\
\subsection*{Funzione Haversine}
La funzione Harvesine, detta anche 'formula dell'emisenoverso', serve per ottenere la distanza in km dal punto di coordinate passato da backend, a quello ottenuto dall'API.